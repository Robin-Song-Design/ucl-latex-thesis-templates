As office space density continues to increase, emergency evacuation efficiency has become a key determining factor for personnel safety. Although research in the fields of crowd dynamics and evacuation simulation is quite extensive \cite{gwynneReviewMethodologiesUsed1999}, most of them focuses on the simulation accuracy of pedestrian models and the impact analysis of isolated parameters. There is still a lack of research on architectural space layout and multi-factor coupling effects.

Therefore this study aims to systematically explore the impact of internal spatial geometric configuration parameters (such as table layout, corridor width, number of exits and door width, etc.) on the efficiency of evacuation of a office space. 

The research will focus on the following questions: 
\begin{itemize}
    \item How does the layout of furnitures affect evacuation efficiency and path formation?
    \item What is the marginal effect of corridor width and exit parameters on evacuation dynamics and the regulatory role of congestion patterns?
    \item How would the gate width before exit influence the formation of the path?
    \item What is the relationship between the evacuation time distribution and the path distribution under different gate width configurations?
\end{itemize}

Through the simulation of two scale office prototypes, we can examine how these internal layout parameters systematically regulate the evacuation process, in order to provide data-driven optimization suggestions for the safety design of office.