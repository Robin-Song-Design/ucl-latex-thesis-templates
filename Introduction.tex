\section{Research Background}
In increasingly dense office spaces, the efficiency of emergency evacuations is directly related to personnel safety. Although considerable research has focused on crowd dynamics and evacuation simulations, the engineering challenge of optimizing evacuation paths and reducing congestion through systematic adjustments of internal layouts, within the constraints of fixed building structures, remains unresolved. 

\section{Research Question}
This study aims to systematically explore the impact of internal spatial geometric configuration parameters (such as table layout, corridor width, number of exits and door width, etc.) on the efficiency of orderly evacuation of building space under the constraints of fixed building structure. The research will focus on the following key issues: How does the layout of tables affect evacuation efficiency and path formation; The marginal effect of corridor width and exit parameters on evacuation dynamics and the regulatory role of congestion patterns; The influence of the Gate waidth before exit on the formation of the path, and the relationship between the evacuation time distribution and the path distribution under different gate width configurations. 

Through the simulation of two scale Office prototypes, it reveals how these internal layout parameters systematically regulate the evacuation process, in order to provide data-driven optimization suggestions for the safety design of office.

