% I may change the way this is done in a future version, 
%  but given that some people needed it, if you need a different degree title 
%  (e.g. Master of Science, Master in Science, Master of Arts, etc)
%  uncomment the following 3 lines and set as appropriate (this *has* to be before \maketitle)
\makeatletter
\renewcommand {\@degree@string} {Master of Science}
\makeatother
% Set course title, comment out to go back to default "University College london"
\makeatletter
\renewcommand {\@course@title} {Architectural Computation}
\makeatother
% Change course title preposition, comment out to go back to default "of"
\makeatletter
\renewcommand {\@course@title@preposition} {in}
\makeatother

% Flags for content selection; 
% They should be before calls to \maketitle, \makedeclaration etc.
% If set to true, download EPS file from: https://imagestore.ucl.ac.uk/imagestore/start/ucl-banners/For%20PRINT?authrequest=1&https=1
\setbool{useuclbanner}{true}
\uclbannerlocation{fig/banner.png}

\title{BARC0141: Built Environment Dissertation}
\subtit{Evaluating Office Layout Effects on Evacuation Time Using Social Force Model}
\author{Robin Song}
\department{The Bartlett School of Architecture}

% You can use this for extra information after the boilerplate declaration.
% Use the \\ macro for new lines. Can be useful if you have to declare
% contributions on papers.
\extradeclaration{}

\maketitle

\begin{abstract} % 300 word limit

With the development of modern office spaces towards higher density, the efficiency of emergency evacuation has become a key factor in ensuring life safety. Existing research mostly focuses on the simulation accuracy of pedestrian models or the independent influence of single parameters, while systematic studies on the variable geometric layout parameters within office spaces and their coupling effects are still insufficient. Therefore, this study systematically explores how internal spatial parameters, such as furniture layout, corridor and door width, and the number of exits, affects evacuation efficiency. The study adopts an agent-based modeling approach and conducts simulation experiments on two office prototypes of different scales in a parametric workflow integrating Rhino and JuPedSim, using the social force model. Quantitative evaluations are made through indicators such as total evacuation time (TET) and congestion hotspots. The research finds that internal layout parameters have a significant nonlinear impact on evacuation efficiency: increasing the width of corridors or exits shows a clear "diminishing marginal benefit" effect; at the same time, there is a strong coupling effect among parameters. When downstream channels are narrow, blindly widening the doors of upstream rooms will instead aggravate congestion and prolong evacuation time. The research conclusion confirms that the internal geometric configuration of office spaces is a key factor affecting evacuation efficiency, and its optimization requires a comprehensive consideration of the thresholds and coupling relationships among parameters, rather than maximizing a single parameter. This study provides parametric design basis for architects, and future research can validate the simulation with real data and consider more dynamic factors.

\textbf{Key Words:} Emergency evacuation, office layout, agent-based modeling, social force model, pedestrian dynamics, parametric design.

\end{abstract}

% \begin{impactstatement}

%     UCL theses now have to include an impact statement. \textit{(I think for REF reasons?)} The following text is the description from the guide linked from the formatting and submission website of what that involves. (Link to the guide: {\scriptsize \url{http://www.grad.ucl.ac.uk/essinfo/docs/Impact-Statement-Guidance-Notes-for-Research-Students-and-Supervisors.pdf}})

%     \begin{quote}
%         The statement should describe, in no more than 500 words, how the expertise, knowledge, analysis,
%         discovery or insight presented in your thesis could be put to a beneficial use. Consider benefits both
%         inside and outside academia and the ways in which these benefits could be brought about.

%         The benefits inside academia could be to the discipline and future scholarship, research methods or
%         methodology, the curriculum; they might be within your research area and potentially within other
%         research areas.

%         The benefits outside academia could occur to commercial activity, social enterprise, professional
%         practice, clinical use, public health, public policy design, public service delivery, laws, public
%         discourse, culture, the quality of the environment or quality of life.

%         The impact could occur locally, regionally, nationally or internationally, to individuals, communities or
%         organisations and could be immediate or occur incrementally, in the context of a broader field of
%         research, over many years, decades or longer.

%         Impact could be brought about through disseminating outputs (either in scholarly journals or
%         elsewhere such as specialist or mainstream media), education, public engagement, translational
%         research, commercial and social enterprise activity, engaging with public policy makers and public
%         service delivery practitioners, influencing ministers, collaborating with academics and non-academics
%         etc.

%         Further information including a searchable list of hundreds of examples of UCL impact outside of
%         academia please see \url{https://www.ucl.ac.uk/impact/}. For thousands more examples, please see
%         \url{http://results.ref.ac.uk/Results/SelectUoa}.
%     \end{quote}
% \end{impactstatement}

% \begin{acknowledgements}
%     First of all, I would like to express my most sincere gratitude to my supervisor, Petros Koutsolampros. From the conception of the topic to the final completion of the thesis, he has always provided me with insightful guidance and valuable suggestions. His rigorous academic attitude and profound professional knowledge helped me overcome many challenges in the research. At the same time, I would also like to thank my technical supervisor, Sherif Tarabishy, who gave me crucial and valuable advice when I was setting up the workflow between Rhino and JuPedSim, which was essential for the technical realization of this research. 

%     I also sincerely want to thank my family and my girlfriend. During my master's studies, they have always been my strongest support. Their understanding and support have been the foundation for me to focus on my studies without any distractions.
% \end{acknowledgements}

\setcounter{tocdepth}{2}
% Setting this higher means you get contents entries for
%  more minor section headers.

\tableofcontents
% \listoffigures
% \listoftables
% \renewcommand\listoflistingscaption{List of source codes}
% \listoflistings

