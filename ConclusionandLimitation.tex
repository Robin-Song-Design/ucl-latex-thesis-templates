\section{Conclusions}
This study systematically compares evacuation performance in office spaces of different scales, revealing the non-linear characteristics of the impact that building design parameters have on crowd evacuation efficiency.

Firstly, evacuation efficiency exhibits a clear pattern of diminishing marginal returns: in the Small Prototype experiment, increasing corridor width from 1.0m to 1.6m results significantly reduced total evacuation time (TET), but over 2.0m, the benefits of further widening become less; the bottleneck experiment before the exit in the Big Prototype shows the same conclusion: after a width of 1.2m it eliminates queuing, and TET is significantly shortened, while the benefits of continued widening decrease. Both of these results support Helbing et al.'s views in 2000 \cite{helbingSimulatingDynamicalFeatures2000}.

Secondly, regarding furniture layout exploration, different layout configurations exhibit distinctly different characteristics under the same corridor width parameter: the layout A ,with linear table arrangement, achieves stable multi-lane flow at a width of 1.6m, while the layout B, with multiple intersections, generates mesh-like low-speed areas under narrow passages, and the layout C with parallel passages shows abnormal fluctuations at specific widths.

For the effects of multi-parameter coupling, the dual-exit configuration of the Small Prototype shows significant differences in improvement effects among different layouts. Layout A effectively eliminates terminal bottlenecks, and Layout B makes full use of upstream advantages, but the improvement for the  Layout C with complex intersection is limited. The Big Prototype experiment reveals coupling effects between room door width and corridor width: when the corridor is narrower, increasing the room door width prolongs evacuation time, with the optimal configuration occurring at a corridor width of 1.4m and a room door width of 1.2m.

When geometric parameters exceed critical thresholds, not only is the average evacuation time improved, but the standard deviation between multiple runs is also significantly reduced, leading to more stable and predictable system behavior. These findings not only support the ideas about critical thresholds and geometric constraints in existing evacuation theories \cite{helbingSimulatingDynamicalFeatures2000,zhouDevelopingDatabasePedestrians2018,santosCRITICALREVIEWEMERGENCY}, but also provide guidance for parameter optimization in practical office design.

\section{Practical Implications and Cautions}

In the actual design optimization process, design can be specifically optimized at different design stages. In the early stages, designers should focus on ensuring that corridor widths reach critical ranges (1.2 - 1.4 meters) to minimize congestion and avoid queue backflow. For example, in the context of a school building, the main corridor connecting multiple classrooms and stairwells would serve as the main evacuation channel. Designers should strive to meet this width range while complying with design regulations and balancing the design budget. While in the later stages, because of structural and cost limitations, designers can focus on exit areas where crowds gather, avoiding situations where exit gates are too narrow or obstructed.

At the same time, designers can also pay attention to the crowd intersections and sharp turns caused by furniture layout, as even slight adjustments can lead to greater optimization effects. Specifically, for example, in the case of workstations, designers may prefer to create neat and uniform layouts, while avoiding designs that resemble the windmill-shaped workstations similar to those in this study, ensuring that the overall layout remains tidy and minimalist. This approach can help minimize intersections in crowd departure areas, directing any evacuation influences toward corridor gathering areas. While focusing on functional areas and circulation routes, designers should also consider the number and distribution of users and reserve some safety space to prevent unexpected intersections and congestion during emergencies due to panic.

More importantly, for situations involving the coupling effects of room doors and corridor doors, designers need to avoid configurations where upstream supply capacity far exceeds downstream release capacity, ensuring that the door width matches corridor gate width. In other words, in a multi-level channel system, the capacities of each level's channels should remain balanced to prevent any one link from becoming a bottleneck. For instance, any layout with nested rooms constitutes a multi-level corridor system. Most corridor doors are actually fire doors, and their fire-blocking function should not be overlooked; however, they can also impact evacuations during emergencies and cause crowding, which needs to be examined based on specific situations.

Design is a wicked problem; there is no absolute true or false, only relatively good and bad; there is no best solution, only better solutions. Architectural designers often have to comprehensively consider functional requirements, structural demands, design budgets, and other aspects, making it difficult to address safety issues like evacuation that arise only in emergencies. This is where the significance of design regulations comes into play. However, when faced with changing geometric layouts and various crowd characteristics and distributions, fixed parameters are hard to generalize to different architectural situations. This necessitates that designers themselves use parametric design tools to simulate real complex scenarios to guide the design process.

\section{Limitations}

\subsection{Model and Parameter Limitations}

The results of this study show that key parameters exhibit threshold response characteristics (such as gate width and corridor width), and small variations near the inherent threshold of the model can lead to disproportionately large effects, especially in large prototypes. The homogenized individual behavior assumed by the social force model may not fully capture individual differences, group behavior, and panic responses in actual evacuations. The static furniture assumption overlooks the impact that potential furniture movement during real evacuations could have on path selection.

\subsection{Generalizability and External Validity}

This study is based on JuPedSim and the social force model, and the limitations of the model's assumptions (such as homogeneous individuals and static furniture) may restrict its applicability in the real world. The scale of the study is relatively limited (18 people in the small prototype and 57 people in the big prototype), which may not reflect the behavioral characteristics of larger-scale evacuations. External validation is needed through empirical evacuation data.

The research results are based on specific office environments and layout types, and the application to other building types such as schools, shopping malls and hospitals, requires further verification. And the experiments focus on one floor layouts, while multi-floor evacuation scenarios may involve additional complexities such as staircases and elevators, which are the main factors affecting evacuation efficiency in high-rise buildings.

\section{Future Work}
Future work should firstly expand the diversity and validation scope of models. Specifically, experiments should be conducted by replacing different models (such as cellular automata models or velocity-based models) and performing systematic comparisons to comprehensively assess the impact of architectural geometric elements on the performance of different models. Meanwhile, real evacuation data should be carried out to validate the simulation results.

To enhance the depth and breadth of the research, more layout and furniture configuration types should be explored, and the number of simulation repetitions should be increased to improve statistical significance. Regarding the complexity of real scenarios, individual differences such as age, physical condition, and familiarity should be considered.  Additionally, we should study the impact of dynamic obstacles, such as furniture, on evacuation paths and congestion points during crowd movement to make the simulation more reflective of real situations.

On a technical level, the integration workflow between Rhino-Grasshopper and JuPedSim can be  packaged as a Grasshopper plugin, improving the interface and data flow, ensuring that parameter transmission is more intuitive, configurations are more flexible, and simultaneously reducing the workload from redundant settings. Moreover, Based on the research conclusions, real-time optimization algorithms can be developed, enabling the system to automatically generate near-optimal layout plans under given constraints, thereby increasing the practical application value of design and validation.