\section{Conclusions}
In this study, we systematically compared Small and Big office prototypes across island layouts A, B, C and varying corridor widths, door/gate widths, and exit configurations. The results show nonlinear coupling between geometry and flow, with scale-dependent sensitivity observed in the Big prototype under narrow corridors. Key findings include:
\section{Limitations}

\subsection{Model and Parameter Sensitivity}

Our results show threshold-like responses to key parameters (e.g., gate width, corridor width). Small changes near intrinsic model thresholds can lead to disproportionately large effects, especially in the Big prototype.
\subsection{Discretization of Parameter Space}

The parameter grid is sampled at discrete values, and behavior near thresholds may differ from interpolations between points. Caution is needed when generalizing beyond tested values.

\subsection{Generalizability and External Validity}

This reasearch is based on JuPedSim and Social Force Model, while Real-world validity may be limited by model assumptions (e.g., homogeneous agents, static furniture). External validation with empirical evacuation data is needed.
\section{Practical Implications and Cautions}

Designers should beware of relying on discrete parameter choices. Critical points near thresholds may drive large changes in performance; validate with site-specific constraints. However, our workflow was built in Rhino, which allows designers to reproduce the simulations easily as needed to verify results or test alternative layouts.

\section{Future Work}
Expand to include alternative models (e.g., cell-based or agent-based with different interaction rules), add external validation with real evacuation data, explore more layouts and furniture configurations, increase repeats, and consider uncertainty quantification