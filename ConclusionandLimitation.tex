\section{Conclusions}
This study systematically compares evacuation performance in office spaces of different scales, revealing the non-linear characteristics of the impact that building design parameters have on crowd evacuation efficiency.

Firstly, evacuation efficiency exhibits a clear pattern of diminishing marginal returns: in the Small Prototype experiment, increasing corridor width from 1.0m to 1.6m results significantly reduced total evacuation time (TET), but over 2.0m, the benefits of further widening become less; the bottleneck experiment before the exit in the Big Prototype shows the same conclusion: after a width of 1.2m it eliminates queuing, and TET is significantly shortened, while the benefits of continued widening decrease. Both of these results support Helbing et al.'s views in 2000 \cite{helbingSimulatingDynamicalFeatures2000}.

Secondly, regarding furniture layout exploration, different layout configurations exhibit distinctly different characteristics under the same corridor width parameter: the layout A ,with linear table arrangement, achieves stable multi-lane flow at a width of 1.6m, while the layout B, with multiple intersections, generates mesh-like low-speed areas under narrow passages, and the layout C with parallel passages shows abnormal fluctuations at specific widths.

For the effects of multi-parameter coupling, the dual-exit configuration of the Small Prototype shows significant differences in improvement effects among different layouts. Layout A effectively eliminates terminal bottlenecks, and Layout B makes full use of upstream advantages, but the improvement for the  Layout C with complex intersection is limited. The Big Prototype experiment reveals coupling effects between room door width and corridor width: when the corridor is narrower, increasing the room door width prolongs evacuation time, with the optimal configuration occurring at a corridor width of 1.4m and a room door width of 1.2m.

When geometric parameters exceed critical thresholds, not only is the average evacuation time improved, but the standard deviation between multiple runs is also significantly reduced, leading to more stable and predictable system behavior. These findings not only support the ideas about critical thresholds and geometric constraints in existing evacuation theories \cite{helbingSimulatingDynamicalFeatures2000,zhouDevelopingDatabasePedestrians2018,santosCRITICALREVIEWEMERGENCY}, but also provide guidance for parameter optimization in practical office design.

\section{Limitations}

\subsection{Model and Parameter Limitations}

The results of this study show that key parameters exhibit threshold response characteristics (such as gate width and corridor width), and small variations near the inherent threshold of the model can lead to disproportionately large effects, especially in large prototypes. The homogenized individual behavior assumed by the social force model may not fully capture individual differences, group behavior, and panic responses in actual evacuations. The static furniture assumption overlooks the impact that potential furniture movement during real evacuations could have on path selection.

\subsection{Generalizability and External Validity}

This study is based on JuPedSim and the social force model, and the limitations of the model's assumptions (such as homogeneous individuals and static furniture) may restrict its applicability in the real world. The scale of the study is relatively limited (18 people in the small prototype and 57 people in the big prototype), which may not reflect the behavioral characteristics of larger-scale evacuations. External validation is needed through empirical evacuation data.

The research results are based on specific office environments and layout types, and the application to other building types such as schools, shopping malls and hospitals, requires further verification.

\section{Practical Implications and Cautions}

During pratical design optimisation, designers can prioritize ensuring that the corridor width reaches the critical range (1.2 - 1.4 m) to avoid queue backflow. At the same time, the merging points, sharp turns and the convergence area before the exit can be focused on, and geometric optimization can be carried out for these areas, because they have the most significant impact on the overall performance. For some cases similar to the coupling effect of room door and corridor gate, designers need to avoid configurations where the upstream supply capacity far exceeds the downstream release capacity, and ensure that the door width matches the width of the corridor gate. In other words, in a multi-level channel system, the capacity of each level of channels should be balanced to prevent any one link from becoming a bottleneck. For complex layouts with multiple intersections, merely increasing the channel width or the number of exits may have limited effects. It may also be necessary to combine local geometric modifications, such as corner optimization and setting guidance measures. The Rhino workflow we have established enables designers to easily reproduce experiments, verify results or test alternative layouts as needed, thus providing a convenient verification method for practical applications.

\section{Future Work}
Future work should firstly expand the diversity and validation scope of models. Specifically, experiments should be conducted by replacing different models (such as cellular automata models or velocity-based models) and performing systematic comparisons to comprehensively assess the impact of architectural geometric elements on the performance of different models. Meanwhile, real evacuation data should be carried out to validate the simulation results.

To enhance the depth and breadth of the research, more layout and furniture configuration types should be explored, and the number of simulation repetitions should be increased to improve statistical significance. Regarding the complexity of real scenarios, individual differences such as age, physical condition, and familiarity should be considered.  Additionally, we should study the impact of dynamic obstacles, such as furniture, on evacuation paths and congestion points during crowd movement to make the simulation more reflective of real situations.

On a technical level, the integration workflow between Rhino-Grasshopper and JuPedSim can be  packaged as a Grasshopper plugin, improving the interface and data flow, ensuring that parameter transmission is more intuitive, configurations are more flexible, and simultaneously reducing the workload from redundant settings. Moreover, Based on the research conclusions, real-time optimization algorithms can be developed, enabling the system to automatically generate near-optimal layout plans under given constraints, thereby increasing the practical application value of design and validation.