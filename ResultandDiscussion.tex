\section{Social Force Model}
In simulations of emergency evacuation scenarios, researchers usually employ modeling methods such as social force models, cellular automata as well as agent-based models to simulate the complexity of crowd behavior. These models have become the primary approach in this field of study.
The social force model is an appealing method because it offers a relatively open and scalable framework. Compared to other models, the social force model can more intuitively explain the interactions between individuals and can replicate complex phenomena in crowded environments, such as speed, density and queuing situations. This model was first introduced by Helbing and Molnár in 1995, attempting to accurately simulate crowd behavior from a microscopic dynamics perspective, analogous to gas dynamics. Its core idea is to view crowd movement as a dynamic process influenced by various "forces".
Specifically, the social force model mainly considers four fundamental effects: (1) pedestrians wish to reach a specific destination; (2) pedestrians need to maintain a certain distance from each other; (3) pedestrians should keep a set distance from obstacles; (4) pedestrians may experience attraction towards specific objects (such as friends). Through the combined effects of these forces, the model can effectively explain the interactions among individuals and replicate complex phenomena in crowded environments.
Many researchers have continuously improved and applied the social force model, further expanding its potential. For example, Zheng and others combined the social force model with neural networks to simulate collective behavior in different scenarios; Seyfried and colleagues made modifications to pedestrian dynamics for qualitative analysis; Parisi and Dorso utilized this model to study the evacuation process in rooms with exits. These studies indicate that the social force model has become an important tool for simulating crowd movement, showing broad application prospects especially in areas like traffic planning and architectural design.

\section{Previous Work}
An important branch of evacuation simulation is the research on physical environmental factors. The key elements of spatial structure include the geometric form of buildings, exit configuration and the distribution of obstacles. The geometric form of buildings involves the shape of rooms, floor height and corridor width, and these parameters directly affect the possibility and efficiency of crowd movement. Exit configuration, including the number, location and width of exits, is a key factor in determining the evacuation speed. In addition, the distribution of obstacles, such as the spatial positions of fixed facilities like tables, chairs and columns, significantly affects the evacuation paths and crowd speeds .

In 2012, Ma et al. conducted experimental research on evacuation in China's skycrappers, focusing on the impact of building physical characteristics on the evacuation process. Peacock and Kuligowski (2004) systematically reviewed occupant movement in buildings and found that physical environment factors, including space configuration, exit placement, and obstacle distribution, significantly affect evacuation efficiency. Koo et al. (2013) compared evacuation strategies in high-rise buildings, particularly examining the paths of disabled individuals under different physical conditions. Their study revealed that the geometry and design of exits directly determine the feasibility and safety of evacuations. In high-density and emergency situations, these physical factors can greatly influence crowd behavior during evacuations. A common conclusion from these studies is that the physical environment is a significant influence on evacuation simulation, necessitating a comprehensive consideration of building shape, exit quantity and location, stair width, and obstacle distribution.

In research on classrooms and office spaces, while exit design is important, an increasing number of studies highlight the significant impact of internal furniture arrangement, desk layout, and corner configurations on evacuation efficiency. In some situations, these internal layout factors can outweigh the influence of exit width or quantity (Liu Parhizgar, 2018;). Helbing et al. specifically emphasized the considerable effect of obstacles, such as desks in a classroom, on evacuation efficiency. These findings indicate that optimizing layout by adjusting internal environmental parameters can significantly alter crowd path choices and congestion hotspots.

\section{Research Gap}
Despite the extensive research on evacuation simulations, there is still a lack of systematic exploration of how internal layout parameters affect evacuation performance in fixed structural settings. 

Firstly, existing studies lack a systematic analysis of how internal layout parameters interact. Most of the studies only focus on single parameter such as number of exits and layout of desks and seats, while ignoring factors including corners, corridor width and door width as well as their interaction, particularly in senarios with multiple exits and neted rooms.

Secondly, there is a lack of research on prototypes with different sizes. The lack of reliable methods for scaling from small and simple spaces to large and complex layout limits the translation of research results into practical applications.

Third, current research primarily focuses on simple scenarios such as classrooms, while systematic studies on the layout of office spaces are relatively scarce. Unlike traditional scenarios like classrooms, office spaces exhibit a high degree of complexity, including densely populated crowds, diverse furniture configurations, and dynamic layouts. The uniqueness of office spaces lies in the relatively fixed initial positions of personnel and the significant flexibility of furniture arrangements. Therefore, changes in the layout of office spaces may significantly impact evacuation efficiency, making them an ideal subject for evacuation simulation research.

Based on the aforementioned research limitations, the innovation of this study lies in: under the fixed structural constraints of office space scenarios, systematically comparing the evacuation performance of different scales and multiple layout variants, and revealing the coupling effects of door width, corridor width, and exit configuration through quantitative analysis. This research not only provides an actionable method for assessing the safety of office spaces but also offers a new methodological perspective for evacuation simulation studies.